\documentclass{article}
\usepackage{graphicx}
\usepackage{color}
\usepackage[outputdir=./build]{minted}
\usepackage[colorlinks = true,
            linkcolor = black,
            urlcolor  = cyan,
            citecolor = cyan,
            anchorcolor = cyan
]{hyperref}
\newcommand{\HWVersion}{0}
\sloppy
\definecolor{lightgray}{gray}{0.5}
\setlength{\parindent}{0pt}
\usepackage{fancyhdr}

\title{MATH 375 - Intro to Numerical Computing \\Homework \HWVersion}
\date{\today}

\pagestyle{fancy}
\fancyhf{}
\rhead{\today}
\lhead{MATH 375 Homework \HWVersion}
\rfoot{Page \thepage}
\begin{document}
\maketitle
\pagebreak[4]
\tableofcontents
\pagebreak[4]
\section*{Basic MATLAB Script}
\addcontentsline{toc}{section}{Basic MATLAB Script}

\begin{par}
This file demonstates how to use the tool and these comments. For more information visit:
\end{par} \vspace{1em}
\begin{par}

\href{https://www.mathworks.com/help/matlab/matlab_prog/marking-up-matlab-comments-for-publishing.html}{MathWorks
docs}

\end{par} \vspace{1em}




\subsection*{Printing out pi}
\addcontentsline{toc}{subsection}{Printing out pi}

\begin{par}
Here we will show what happens when we print $\pi$ in different ways
\end{par} \vspace{1em}
\begin{minted}{matlab}
pi
disp(pi)
fprintf("%.69f\n", pi);
\end{minted}

        \color{lightgray} \begin{verbatim}
ans =

    3.1416

    3.1416

3.141592653589793115997963468544185161590576171875000000000000000000000
\end{verbatim} \color{black}
    

\subsection*{Conclusion}
\addcontentsline{toc}{subsection}{Conclusion}

\begin{par}
As you probably noticed, the result of running the previous section is displayed right below the code chunk. Now isnt that so nice?
\end{par} \vspace{1em}
\begin{minted}{matlab}
"Good bye"
disp("Good bye")
% oh and by the way you can still add comments
fprintf("%.69f\n", "Good bye");
\end{minted}

        \color{lightgray} \begin{verbatim}
ans = 

    "Good bye"

Good bye
NaN
\end{verbatim} \color{black}
\section*{Advanced MATLAB Script}
\addcontentsline{toc}{section}{Advanced MATLAB Script}

\begin{par}
We will be going over more advanced matlab topics and how they are displayed
\end{par} \vspace{1em}




\subsection*{Making a table}
\addcontentsline{toc}{subsection}{Making a table}

\begin{par}
Not much to see here, fairly simple
\end{par} \vspace{1em}
\begin{minted}{matlab}
x = linspace(0, 25, 20);
calculated_cosine = cosine(x);

% this table is output in markdown form so you should probably go use a
% markdown renderer if it needs to be pretty
fprintf("| x | cos(x) |\n");
fprintf("|---|---|\n");

for i = 1:length(x)
    fprintf("| %.4f | %.4f |\n", x(i), calculated_cosine(i));
end
\end{minted}


\subsection*{Making a chart}
\addcontentsline{toc}{subsection}{Making a chart}

\begin{par}
Lets chart it!
\end{par} \vspace{1em}
\begin{minted}{matlab}
plot(x,calculated_cosine);
\end{minted}

\includegraphics [width=4in]{./HW0/html/Demostration02_01.eps}


\subsection*{Different plotting styles}
\addcontentsline{toc}{subsection}{Different plotting styles}

\begin{par}
That was trash, lets see if log scale is better
\end{par} \vspace{1em}
\begin{minted}{matlab}
yyaxis left
plot(x,calculated_cosine);

% we will add one so that the cosine is always positive and thus can be
% plotted
yyaxis right
semilogy(x,1+calculated_cosine);
\end{minted}

\includegraphics [width=4in]{./HW0/html/Demostration02_02.eps}


\subsection*{Add more x values}
\addcontentsline{toc}{subsection}{Add more x values}

\begin{par}
maybe if we did a bigger \texttt{linspace} it would work
\end{par} \vspace{1em}
\begin{minted}{matlab}
x = linspace(0, 25, 2000);
calculated_cosine = cosine(x);
\end{minted}


\subsection*{Replotting}
\addcontentsline{toc}{subsection}{Replotting}

\begin{par}
Now we will re-plot with the new x values
\end{par} \vspace{1em}
\begin{minted}{matlab}
yyaxis left
plot(x,calculated_cosine);

% we will add one so that the cosine is always positive and thus can be
% plotted
yyaxis right
semilogy(x,1+calculated_cosine);
\end{minted}

\includegraphics [width=4in]{./HW0/html/Demostration02_03.eps}


\subsection*{Functions}
\addcontentsline{toc}{subsection}{Functions}

\begin{par}
Here follows some helper functions.
\end{par} \vspace{1em}
\begin{par}

\subsubsection*{cosine}
\addcontentsline{toc}{subsubsection}{cosine}

\end{par} \vspace{1em}
\begin{par}
We include that since we want it to show up underneath the functions subsection. Anyway this function is a wrapper for the cosine function. Totally useless but I included it for demo purposes
\end{par} \vspace{1em}
\begin{minted}{matlab}
function Y = cosine(x)
    Y = cos(x);
end
\end{minted}

        \color{lightgray} \begin{verbatim}| x | cos(x) |
|---|---|
| 0.0000 | 1.0000 |
| 1.3158 | 0.2523 |
| 2.6316 | -0.8727 |
| 3.9474 | -0.6926 |
| 5.2632 | 0.5233 |
| 6.5789 | 0.9566 |
| 7.8947 | -0.0407 |
| 9.2105 | -0.9771 |
| 10.5263 | -0.4522 |
| 11.8421 | 0.7490 |
| 13.1579 | 0.8301 |
| 14.4737 | -0.3302 |
| 15.7895 | -0.9967 |
| 17.1053 | -0.1726 |
| 18.4211 | 0.9096 |
| 19.7368 | 0.6315 |
| 21.0526 | -0.5910 |
| 22.3684 | -0.9297 |
| 23.6842 | 0.1220 |
| 25.0000 | 0.9912 |
\end{verbatim} \color{black}
    

\subsection*{Conclusion}
\addcontentsline{toc}{subsection}{Conclusion}

\begin{par}
Thanks for playing. Remember to like an subscribe.
\end{par} \vspace{1em}
\end{document}